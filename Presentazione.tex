\documentclass{beamer}
\usetheme{Madrid}
% Abilita il supporto alle immagini
\usepackage{graphicx}
%Path relative to the main .tex file 
\graphicspath{ {./images/} }
\usepackage{fancybox}

% Informazioni da includere nella pagina del titolo:
\title[DCC performance in IEEE 802.11p] % opzionale
{On the performance of Decentralized Congestion Control in a real IEEE 802.11p testbed}

\fontsize{11}{13}\selectfont \author{\textbf{Studente: Antonio Solida - 178507}}

\institute[] % opzionale
{
    \fontsize{11}{13}\selectfont \textbf{Relatore: Prof. Carlo Augusto Grazia}
    \and
    Esame di Automotive Connectivity\\
    Corso di Laurea Magistrale in Ingegneria Informatica\\
    Percorso "\textit{Cloud \& Cybersecurity}"
    \and
    Dipartimento di Ingegneria "Enzo Ferrari"\\
    Università degli studi di Modena e Reggio Emilia
}

\date[17 ottobre 2024] % opzionale
{Modena, 17 ottobre 2024}

\begin{document}

\frame{\titlepage}

\begin{frame}
\frametitle{Introduzione}
\centering
\begin{block}{Keywords}
    \begin{itemize}
        \item Reti \textbf{VANET} (\textit{Vehicular Ad-Hoc Networks}) \& \textbf{DSRC} (\textit{Dedicated Short-Range Communication}).
        \item IEEE \textbf{802.11p} \& ETSI \textbf{ITS} (\textit{Intelligent Transport System}).
        \item \textbf{CAM} (\textit{Community Awareness Message}).
    \end{itemize}
\end{block}

\begin{columns}
    % Column 2    
    \begin{column}{0.5\textwidth}
        \begin{figure}[h!]
            \centering
            \includegraphics[width=1\textwidth]{vanet.png}
            \label{fig:vanet}
        \end{figure}
    \end{column}
    \begin{column}{0.5\textwidth}
        \begin{figure}[h!]
            \centering
            \includegraphics[width=1\textwidth]{routing_vanet.jpeg}
            \label{fig:obu_rsu}
        \end{figure}
    \end{column}
\end{columns}
\end{frame}

\begin{frame}
    \frametitle{Problema: congestione del canale}
    In ambienti con più veicoli, come una strada trafficata, l'accumulo di informazioni trasmesse può portare a una \alert{seria congestione del canale}, compromettendo la comunicazione.

    \begin{block}{Possibili cause}
        \begin{itemize}
            \item \textbf{CAM inviati periodicamente:} \\
            I Cooperative Awareness Messages (CAM) vengono trasmessi in un intervallo di tempo compreso tra 0,1 e 1 secondo.
            
            \item \textbf{Invio di DENM:} \\
            Possibile invio di Decentralized Environmental Notification Messages (DENM) per comunicazioni specifiche.
            
            \item \textbf{Altre trasmissioni sulla stessa banda:} \\
            La presenza di altre comunicazioni e disturbi sulla stessa banda - \textit{5,9 GHz (5,850 - 5,925 GHz)} - può influire sulla qualità del segnale.
        \end{itemize}
    \end{block}
    
\end{frame}

\begin{frame}
    \frametitle{Soluzione: DCC con Transmit queues?}
    \centering
    \includegraphics[width=0.5\textwidth]{dcc_edca.jpg}
    \begin{itemize}
        \item \textbf{Decentralized Congestion Control (DCC):} Meccanismo standardizzato dell'ETSI per controllare la congestione; sono stati introdotti parametri aggiuntivi basati sulle categorie di accesso per gestire le diverse priorità e requisiti di latenza delle comunicazioni veicolari (\textit{Access Layer}).
        \item \textbf{Classificazione del Traffico:} Ogni Access Category è associata a specifiche caratteristiche di traffico, consentendo una gestione più fine delle risorse di comunicazione.
    \end{itemize}
\end{frame}


\begin{frame}
    \frametitle{Descrizione testbed}
    \centering
    % Aggiunta dell'immagine del diagramma
    \begin{minipage}{0.6\textwidth}
        \includegraphics[width=\textwidth]{topology.png}
    \end{minipage}
    \hfill
    % Aggiunta dell'immagine della scheda e del laptop
    \begin{minipage}{0.35\textwidth}
        \centering
        % Immagine della scheda Rock 3A
        \includegraphics[width=\textwidth]{ROCK_3A.png}
        \vspace{0.5cm}
        % Immagine del laptop
        \includegraphics[width=\textwidth]{ROCK_3A.png}
    \end{minipage}
\end{frame}

\begin{frame}
    \frametitle{Test performance flussi TCP}
    Dispositivi suddivisi in due gruppi:
    \centering
    \includegraphics[width=0.7\textwidth]{Rock scheme.png}
    \begin{itemize}
        \item \textit{Gruppo "CAM"}: due dispositivi che simuleranno un ambiente in cui sono presenti uno svariato numero di veicoli che porteranno il canale di trasmissione a congestionarsi, in maniera parziale (\textit{un CAM ogni 5 ms}) o completa (\textit{flooding UDP}).
        \item \textit{Gruppo "Transmission"}: due dispositivi che comunicheranno mediante due flussi TCP in parallelo mediante l'utilizzo del comando iPerf. Due casi visti: con e senza QoS.
    \end{itemize}
\end{frame}

\begin{frame}
    \frametitle{Trasmissione in ambiente non congestionato}
    
    \begin{minipage}{0.45\textwidth}
        \textbf{Quality of Service assente}\\
        \textit{Throughput medio Stream ID 1}: 3.49 Mbps\\
        \textit{Throughput medio Stream ID 2}: 3.47 Mbps\\
        \vspace{1cm}
        
        \textbf{Quality of Service presente}\\
        \textit{Throughput medio Stream ID 1 (AC\_VO)}: 7.96 Mbps\\
        \textit{Throughput medio Stream ID 2 (AC\_BK)}: 0.61 Mbps\\
    \end{minipage}
    \hfill
    \begin{minipage}{0.5\textwidth}
        \centering
        \begin{minipage}{\textwidth}
            % Primo grafico (Quality of Service assente)
            \includegraphics[width=\textwidth]{t1_c0_main.png}
            \vspace{0.5cm}
        \end{minipage}
        \begin{minipage}{\textwidth}
            % Secondo grafico (Quality of Service presente)
            \includegraphics[width=\textwidth]{t2_c0_main.png}
        \end{minipage}
    \end{minipage}

\end{frame}


\begin{frame}
    \frametitle{Trasmissione in ambiente parzialmente congestionato}
    
    \begin{minipage}{0.45\textwidth}
        % Quality of Service assente
        \textbf{Quality of Service assente}\\
        \textit{Throughput medio Stream ID 1}: 3.13 Mbps\\
        \textit{Throughput medio Stream ID 2}: 3.13 Mbps\\
        
        \vspace{1cm}
        
        % Quality of Service presente
        \textbf{Quality of Service presente}\\
        \textit{Throughput medio Stream ID 1 (AC\_VO)}: 6.73 Mbps\\
        \textit{Throughput medio Stream ID 2 (AC\_BK)}: 0.77 Mbps\\
    \end{minipage}
    \hfill
    \begin{minipage}{0.5\textwidth}
        \centering
        \begin{minipage}{\textwidth}
            % Primo grafico (Quality of Service assente)
            \includegraphics[width=\textwidth]{t1_c1_main.png}
            \vspace{0.5cm}
        \end{minipage}
        \begin{minipage}{\textwidth}
            % Secondo grafico (Quality of Service presente)
            \includegraphics[width=\textwidth]{t2_c1_main.png}
        \end{minipage}
    \end{minipage}

\end{frame}

\begin{frame}
    \frametitle{Trasmissione in ambiente totalmente congestionato}
    
    \begin{minipage}{0.45\textwidth}
        % Quality of Service assente
        \textbf{Quality of Service assente}\\
        \textit{Throughput medio Stream ID 1}: 1.14 Mbps\\
        \textit{Throughput medio Stream ID 2}: 1.19 Mbps\\
        
        \vspace{1cm}
        
        % Quality of Service presente
        \textbf{Quality of Service presente}\\
        \textit{Throughput medio Stream ID 1 (AC\_VO)}: 6.19 Mbps\\
        \textit{Throughput medio Stream ID 2 (AC\_BK)}: 0.30 Mbps\\
    \end{minipage}
    \hfill
    \begin{minipage}{0.5\textwidth}
        \centering
        \begin{minipage}{\textwidth}
            % Primo grafico (Quality of Service assente)
            \includegraphics[width=\textwidth]{t1_c2_main.png}
            \vspace{0.5cm}
        \end{minipage}
        \begin{minipage}{\textwidth}
            % Secondo grafico (Quality of Service presente)
            \includegraphics[width=\textwidth]{t2_c2_main.png}
        \end{minipage}
    \end{minipage}

\end{frame}


\begin{frame}
    \frametitle{Considerazioni}
    \centering
    \includegraphics[width=0.9\textwidth]{throughput_bar.png}
    \begin{block}{Risultati riscontrati}
        \begin{itemize}
            \item Maggior throughput nel flusso a priorità maggiore.
            \item Maggiore efficienza nello sfruttamento della banda.
        \end{itemize}
    \end{block}
\end{frame}

\begin{frame}
    \frametitle{Conclusioni finali}

    \begin{enumerate}
        \item \textbf{Saturazione del Canale:} In assenza di meccanismi di gestione, il numero crescente di dispositivi può saturare il canale, riducendo la qualità del servizio, soprattutto in scenari urbani.

        \item \textbf{Classi di Priorità:} Implementazione di classi di priorità per i messaggi trasmessi, migliorando l'uso delle risorse di rete e garantendo banda per comunicazioni critiche.

        \item \textbf{Risultati Sperimentali:} L'adozione di code di priorità ha dimostrato di aumentare significativamente la disponibilità di banda e ridurre i tempi di attesa per messaggi critici.

        \item \textbf{Implicazioni Future:} Necessità di valutare l'integrazione di datarate più elevati e modulazioni più efficienti per migliorare le performance di rete.
    \end{enumerate}

\end{frame}

\begin{frame}
    \frametitle{Grazie per l'attenzione!}

    \begin{center}
        \textbf{Grazie per l'attenzione!} \\[1em]
        A disposizione per eventuali domande e dubbi.
    \end{center}

\end{frame}

\end{document}
